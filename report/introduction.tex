\section{Introduction}
\paragraph{}
Twitter bots are software programs which generate tweets and push them out to twitter. In the recent past they have gained a lot of popularity. These bots are programmed to generate tweets using different techniques and for different purposes. They present interesting challenges, enticing large organizations and young students alike. Tweet generation takes into account not only the syntactic grammar of a language, but also the context, and thus it makes the task quite complicated.
\paragraph{}
Tweet generation can be broadly divided into two major categories - those that use a constructive approach with the help of grammars and rule rewriting systems, and those that use a data driven approach employing one or more machine learning techniques. While grammars tend to be easier to comprehend, coming up with a grammar that has all the rules needed for meaningful generation of tweets, is a very challenging task. On the other hand, machine learning provides an easier way to bypass the black art of grammar design but comes at the cost of being quite complex to implement.
\paragraph{}
This presents a few interesting problems. Can we augment the constructive approaches (the ones that use grammars) to ease their application to a process such as tweet generation? Or, can we reduce the overhead of data driven approaches to allow their use for processes such as tweet generation? If neither of these options are viable, can we come up with some kind of a combination of the two approaches to use existing data to help come up with grammars that can be analyzed and understood by people, and subsequently used, to generate tweets? 
\paragraph{}
This paper describes just such a hybrid system which utilizes the simplicity of representation of a grammar and the power of machine learning to generate tweets. The machine learning in this system is based on the popularity of tweets in a data set, and the idea, that, emulating popular tweets will help generate better and more meaningful tweets.
\paragraph{}
The paper starts off with describing our approach to tweet generation, followed by some observations and then ends with a discussion of our results and future work.