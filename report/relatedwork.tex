\section{Related Work}
There has been quite some work on twitter bots in the past. From big companies to individual researchers, people from different fields have contributed to the development of twitter bots. The technologies employed range from using predefined grammars to utilizing machine learning to make the bot learn the art of tweet generation using natural language processing ,artificial neural networks and deep learning.

\paragraph{}
In 2015, a tool called Tracery\cite{tracery} was developed. It uses grammars to generate sentences in English. Since its development, lots of developers have used this tool to generate tweets in different domains. The simplicity of context free grammar along with the control that this tool provides, makes it easy to generate good, meaningful tweets. Tracery works like a rule based system, where one writes down all the rules, before hand, and then the system works on those rules to generate sentences. The downside of such a system is the effort required to write down these rule sets along with the limitation of generation which would limit the output to the ruleset defined. It does not provide a way to optimize the structure of sentences by user feedback or any other kind of learning mechanism.

\paragraph{}
On the other hand, Microsoft released its own Twitter bot called Tay in 2016. This bot used machine learning to learn the art of tweet generation, and then generated meaningful and syntactically sound tweets. It was one of the best bots at generating domain independent tweets that followed the linguistic constraints and were semantically sound. However, training such a bot required not only a huge corpus of data, but also a lot of time and a lot of effort to construct complex neural networks.

\paragraph{}
We think that genetic algorithms offer a fair middle ground between the power and complexity of machine learning and the simplicity and instinctiveness of grammars. Hence, we provide an implementation that uses genetic algorithms along with Tracery for tweet generation. We strongly believe that genetic algorithms are simple enough to overcome the amount of effort needed by machine learning, while being complex enough to provide better learning and feedback in the tweet generation process than grammars can, on their own. To our knowledge, this is a novel combination and has not been explored much by anyone.